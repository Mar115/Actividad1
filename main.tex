\documentclass{article}
\usepackage[utf8]{inputenc}
\usepackage[spanish]{babel}
\usepackage{listings}
\usepackage{graphicx}
\graphicspath{ {images/} }
\usepackage{cite}

\begin{document}

\begin{titlepage}
    \begin{center}
        \vspace*{1cm}
            
        \Huge
        \textbf{Clase de repositorios}
            
        \vspace{0.5cm}
        \LARGE
        Informática II
            
        \vspace{1.5cm}
            
        \textbf{Marcela Flórez Orellano }
            
        \vfill
            
        \vspace{0.8cm}
            
        \Large
        Despartamento de Ingeniería Electrónica y Telecomunicaciones\\
        Universidad de Antioquia\\
        Medellín\\
        9 de Marzo de 2021
            
    \end{center}
\end{titlepage}

\tableofcontents
\newpage
\section{Solución al desafío}\label{intro}
\begin{enumerate}
\item Con su mano derecha, deslice suavemente la hoja sobre la superficie plana de la mesa hacia la derecha, a una distancia aproximada de las tarjetas de tres dedos de su mano.
\begin{itemize}
\item Los dedos deben estar en posición vertical sobre la mesa.
\end{itemize}
\item Levante las dos tarjetas por uno de sus lados cortos, con su dedo pulgar y dedo índice de la mano derecha en forma de pinza, teniendo en cuenta que las caras de las tarjetas no queden de frente a usted, de manera que su dedo pulgar quede en la cara de la tarjeta izquierda y el dedo índice quede en la cara de la tarjeta derecha.
\begin{itemize}
\item Haciendo referencia a la cara de las tarjetas como la parte frontal o la parte más ancha.
\end{itemize}
\item Ubique las tarjetas juntas y sin soltarlas sobre el centro de la superficie de la hoja, haciendo que los lados cortos y libres de las tarjetas queden en contacto con la hoja.
\item Conservando la misma posición del paso anterior, levante suavemente de la superficie de la hoja la tarjeta del lado izquierdo con el dedo pulgar, sin dejar de sostener con su dedo índice la tarjeta derecha.  
\item Separe poco a poco con su dedo pulgar la tarjeta del lado izquierdo hacia la izquierda, y al mismo tiempo, incline levemente con su índice la tarjeta derecha hacia la izquierda. Hasta formar con los lados largos de las tarjetas y la hoja la figura de un triángulo que mire hacia usted.
\item Realizar el paso número cinco hasta que las tarjetas se encuentren en equilibrio, para esto se puede ayudar de su dedo medio, sin dejar de mantener la posición de pinza con los dedos índice y pulgar.
\item Logrado el paso anterior, aleje cuidadosamente la mano derecha de las tarjetas sin que se caigan las mismas.
\end {enumerate}

\newpage

\section{Inclusión de imágenes} \label{imagenes}
Se presenta el contenido de la carpeta de imágenes.
\begin{itemize}
\item La figura 1 es el estado inicial, en este se encuentran dos tarjetas ubicadas una sobre la otra, alineadas en la misma orientación, debajo de una hoja de papel. 
\item La figura 2 es el estado final, en este las tarjetas se encuentran en el centro de la hoja de papel en forma de pirámide.

\end{itemize}

\begin{figure}[h]
\includegraphics[width=4cm]{posición A.jpeg}
\centering
\caption{Estado inicial}
\label{fig:posición A}

\vspace*{1cm}

\includegraphics[width=4cm]{posición B.jpeg}
\centering
\caption{Estado final}
\label{fig:posición B}

\end{figure}
\end{document}